%\documentclass[11pt,letterpaper]{article}
\documentclass[letterpaper,amsmath,amssymb,prb,preprint,12pt]{revtex4-1}% Physical Review B
\usepackage{dcolumn}% Align table columns on decimal point
%\usepackage{graphicx,subfigure} %Include figure files
%\usepackage{bm} %bold math
%\usepackage{epstopdf} %handle .eps figures
%\usepackage{mciteplus} %magic
%\usepackage{gensymb} %Generic symbols for both text and math mode
\usepackage[version=4]{mhchem}
% note: don't use \usepackage{chemmacros}. its arrow is prettier, but it does not play well with revtex.
% revtex sets all the defaults to 0 so you'll need
%\setchemformula{bond-length=.5833em,plus-space=.6em,stoich-space=.1383em,adduct-space=.1333em,charge-hshift=.25em}

%\usepackage[margin=1in]{geometry} %one inch margins
%\linespread{2} %double space
%\setlength{\parskip}{8pt} %spacing after paragraph


\begin{document}
\setcounter{page}{0}
\raggedbottom

% 1. TITLE PAGE: Must contain your name, your TA’s name, the date of the
% experiment’s performance, the date the lab report was submitted, and the title of the
% experiment.

\begin{titlepage}
\begin{center}

\vspace*{6cm}
\Large
\textbf{My CHEM 2312 Lab Report Template}

\vspace{0.5cm}
\large
Jennings Zhang
\vspace{1.5cm}

CHEM 2312\\
TA: Jiansong Cai\\
Experiment Date: January 28, 2019\\
Report Date: February 4, 2019

\end{center}
\thispagestyle{empty} %hide page number 0
\end{titlepage}

%-------------------------------------------------------------------------------
%                End of Title Page
%-------------------------------------------------------------------------------

\section{Introduction}

This template took two hours to make... Well, it's done now. And it looks pretty good. Rev\TeX is really annoying in how it throws a warning for no reason. It be like that sometimes, it really do. This first page has a few example stuff that might be useful to copy and paste.

\subsection{Example Stuff}

\subsubsection{From NGRD}

\setlength{\parskip}{8pt} %spacing after paragraph

Below is a mathy formula.

\[\Delta\rho(r)=\rho(r)_{I}-\rho(r)_{M}-\rho(r)_{S}\]

And here is Table \ref{paw}.

\begin{table}[h]
\caption{Atomic and Electronic Parameters Used to Generate PAW Potentials}
\begin{tabular}{l@{\qquad}c@{\qquad}c}
\hline
Element&Electron Configuration&Radius Cut-off (a.u.)\\
\hline
\hline
C & [He]2s$^{2}$2p$^{2}$ & 1.3 \\
N & [He]2s$^{2}$2p$^{3}$ & 1.3 \\
O & [He]2s$^{2}$2p$^{4}$ & 1.41 \\
Si & [Ne]3s$^{2}$3p$^{2}$ & 1.5 \\
P & [He]2s$^{2}$2p$^{6}$3s$^{2}$3p$^{3}$ & 1.51 \\
\hline
\end{tabular}
\label{paw}
\end{table}

\subsubsection{More Examples}

A chemical reaction from the caffeine lab below.

\ce{C6H5-OH + 2Na+ + CO3^2- -> C6H5ONa + NaHCO3}

The equation that follows is set in a wide format, i.e., it spans the full page. 
The wide format is reserved for long equations
that cannot easily be set in a single column:
\begin{widetext}
\begin{equation}
{\cal R}^{(\text{d})}=
 g_{\sigma_2}^e
 \left(
   \frac{[\Gamma^Z(3,21)]_{\sigma_1}}{Q_{12}^2-M_W^2}
  +\frac{[\Gamma^Z(13,2)]_{\sigma_1}}{Q_{13}^2-M_W^2}
 \right)
 + x_WQ_e
 \left(
   \frac{[\Gamma^\gamma(3,21)]_{\sigma_1}}{Q_{12}^2-M_W^2}
  +\frac{[\Gamma^\gamma(13,2)]_{\sigma_1}}{Q_{13}^2-M_W^2}
 \right)\;. 
 \label{eq:wideeq}
\end{equation}
\end{widetext}
This is typed to show how the output appears in wide format.
(Incidentally, since there is no blank line between the \texttt{equation} environment above 
and the start of this paragraph, this paragraph is not indented.)

Table \ref{tab:table1} style is recommended by apssamp.tex, but I prefer table \ref{tab:table2}.

\begin{table}[t]%The best place to locate the table environment is directly after its first reference in text
\caption{\label{tab:table1}%
A table that fits into a single column of a two-column layout. 
Note that REV\TeX~4 adjusts the intercolumn spacing so that the table fills the
entire width of the column. Table captions are numbered
automatically. 
This table illustrates left-, center-, decimal- and right-aligned columns,
along with the use of the \texttt{ruledtabular} environment which sets the 
Scotch (double) rules above and below the alignment, per APS style.
}
\begin{ruledtabular}
\begin{tabular}{lcdr}
\textrm{Left\footnote{Note a.}}&
\textrm{Centered\footnote{Note b.}}&
\multicolumn{1}{c}{\textrm{Decimal}}&
\textrm{Right}\\
\colrule
1 & 2 & 3.001 & 4\\
10 & 20 & 30 & 40\\
100 & 200 & 300.0 & 400\\
\end{tabular}
\end{ruledtabular}
\end{table}

\begin{table}[h]%The best place to locate the table environment is directly after its first reference in text
\caption{\label{tab:table2}%
A table that fits into a single column
}
\begin{tabular}{ l @{\qquad} c }
\toprule
\textrm{Material}&
\textrm{Bond Length}\\
\colrule
2D Aluminum & \(\thicksim\)2.54 \text{\AA}\\
Cycloaluminum helides & 2.643 \text{\AA}, 2.587 \text{\AA}\\
Aluminum tetramers & 2.773 \text{\AA}, 2.767 \text{\AA}\\
\botrule
\end{tabular}
\end{table}

% See fig \ref{sm}.

% \begin{figure}[h!]
% \includegraphics[width=\textwidth]{IMG_3610.PNG}
% \caption{MinutePhysics}
% \label{sm}
% \end{figure}

\setlength{\parskip}{0pt}
\newpage

%-------------------------------------------------------------------------------
%                End of example content
%-------------------------------------------------------------------------------


\section{Objective}

% 2. OBJECTIVE: Briefly explain the goal of the experiment in your own words. It
% should detail how the goal will be met and what techniques will be utilized.

2. OBJECTIVE: Briefly explain the goal of the experiment in your own words. It
should detail how the goal will be met and what techniques will be utilized.

\section{Reaction Equations}

% 3. REACTION EQUATIONS: All reagents, products, and reaction conditions should
% be shown in a balanced equation. Do not include a reaction mechanism unless it is
% specifically requested by your TA. This section may be written-in by hand.

3. REACTION EQUATIONS: All reagents, products, and reaction conditions should
be shown in a balanced equation. Do not include a reaction mechanism unless it is
specifically requested by your TA. This section may be written-in by hand.

\section{Theoretical Yield}

% 4. THEORETICAL YIELD: Calculate your theoretical yield by way of the amount of
% limiting reagent, which should be specifically identified, actually used (i.e. if your
% protocol says to use 0.50g of benzil, your limiting reagent, and you measure out
% 0.49g of benzil, use 0.49g to calculate theoretical yield). Show all your work. This
% section may also be written-in by hand.

4. THEORETICAL YIELD: Calculate your theoretical yield by way of the amount of
limiting reagent, which should be specifically identified, actually used (i.e. if your
protocol says to use 0.50g of benzil, your limiting reagent, and you measure out
0.49g of benzil, use 0.49g to calculate theoretical yield). Show all your work. This
section may also be written-in by hand.

\section{Procedure}

% 5. PROCEDURE: The procedure must be in your own words (do not just copy the
% protocol) and should be a short paragraph detailing the general order of operations.

5. PROCEDURE: The procedure must be in your own words (do not just copy the
protocol) and should be a short paragraph detailing the general order of operations.

\section{Observations}

% 6. OBSERVATIONS: What you actually did in lab and what you observed (color
% changes, etc). It should be written in third person past-tense (no “I”, “We”, or
% “They”). Deviations from the protocol and how any problems were corrected should
% also be noted.

6. OBSERVATIONS: What you actually did in lab and what you observed (color
changes, etc). It should be written in third person past-tense (no “I”, “We”, or
“They”). Deviations from the protocol and how any problems were corrected should
also be noted.

\section{Characterization}

% 7. CHARACTERIZATION: Percent yield (may be hand-written, show all work), the
% physical description of the final product, and the results and analysis (purity,
% identification) of any characterization (melting-point, FTIR, NMR, TLC) should all
% be included in this section.

7. CHARACTERIZATION: Percent yield (may be hand-written, show all work), the
physical description of the final product, and the results and analysis (purity,
identification) of any characterization (melting-point, FTIR, NMR, TLC) should all
be included in this section.

\section{Conclusion}

% 8. CONCLUSION: This section should include an analysis of the generated data,
% whether or not the results were as expected, and whether there were incidents during
% the experiment that could have had negative effects on the yield or purity of the
% product. Don’t repeat previous sections unless it is necessary for clarity.

8. CONCLUSION: This section should include an analysis of the generated data,
whether or not the results were as expected, and whether there were incidents during
the experiment that could have had negative effects on the yield or purity of the
product. Don’t repeat previous sections unless it is necessary for clarity.

% \newpage

% \clearpage %force bibliography to be at end
% \bibliography{ref}
% \bibliographystyle{apsrevM}

\end{document}
